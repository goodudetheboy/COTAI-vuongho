\documentclass[12pt]{article}
\usepackage{graphicx}
\usepackage{mathtools}
\usepackage{calc}
\usepackage{amssymb}
\setlength\parindent{0pt} %% Do not touch this


%% -----------------------------
%% TITLE
%% -----------------------------
\title{Homework Week \#4} %% Assignment Title

\author{Ho Chi Vuong\\ %% Student name
AI Math Foundations: Abstract Vector Spaces\\ %% Code and course name
\textsc{Center of Talent in AI}
}

\date{\today} %% Change "\today" by another date manually
%% -----------------------------
%% -----------------------------

%% %%%%%%%%%%%%%%%%%%%%%%%%%
\begin{document}
   
%% %%%%%%%%%%%%%%%%%%%%%%%%%
\maketitle

% --------------------------
% Start here
% --------------------------

% %%%%%%%%%%%%%%%%%%%
\section*{Section P}
% %%%%%%%%%%%%%%%%%%%


% %%%%%%%%%%%%%%%%%%%
\subsection*{P3.1}
% %%%%%%%%%%%%%%%%%%%
\textbf{Cauchy-Schwarz inequality: $|\left \langle v,w \right \rangle| \leq \Vert v\Vert\Vert w\Vert$}

Beside the solution in Terence Tao's note, we can prove it intuitively as follows. We have the following:
$$|\left \langle v,w \right \rangle| \leq \Vert v\Vert\Vert w\Vert$$
$$\Leftrightarrow (\langle v,w \rangle)^2 \leq \Vert v\Vert^2\Vert w\Vert^2$$
$$\Leftrightarrow (\Vert v\Vert\Vert w\Vert cos(v,w))^2 \leq \Vert v\Vert^2\Vert w\Vert^2$$
$$\Leftrightarrow \Vert v\Vert^2\Vert w\Vert^2 cos(v,w)^2 \leq \Vert v\Vert^2\Vert w\Vert^2$$
As $-1 \leq cos(v,w) \leq 1$, we have proved the above inequality.


% %%%%%%%%%%%%%%%%%%%
\subsection*{P3.2}
% %%%%%%%%%%%%%%%%%%%
\textbf{Triangle inequality: $\Vert v\Vert-\Vert w\Vert\leq \Vert v+w\Vert\leq \Vert v\Vert+\Vert w\Vert$}\\
We have the following:
$$\Vert v\Vert-\Vert w\Vert\leq \Vert v+w\Vert\leq \Vert v\Vert+\Vert w\Vert$$
$$\Leftrightarrow \sqrt{\langle v,v \rangle} -\sqrt{\langle w,w \rangle} \leq \sqrt{\langle v+w, v+w \rangle} \leq \sqrt{\langle v,v \rangle} +\sqrt{\langle w,w \rangle}$$
$$\Leftrightarrow \langle v,v \rangle -2\sqrt{\langle v,v \rangle\langle w,w \rangle} +\sqrt{\langle w,w \rangle} \leq \langle v, v+w \rangle + \langle w, v+w \rangle \leq \langle v,v \rangle +2\sqrt{\langle v,v \rangle\langle w,w \rangle} +\sqrt{\langle w,w \rangle}$$
$$\Leftrightarrow \langle v,v \rangle -2\sqrt{\langle v,v \rangle\langle w,w \rangle} +\langle w,w \rangle \leq \langle v, v \rangle + \langle v, w \rangle + \langle w, v \rangle + \langle w, w \rangle\leq \langle v,v \rangle +2\sqrt{\langle v,v \rangle\langle w,w \rangle} +\langle w,w \rangle$$
$$\Leftrightarrow -2\sqrt{\langle v,v \rangle\langle w,w \rangle} \leq \langle v, w \rangle + \langle w, v \rangle \leq 2\sqrt{\langle v,v \rangle\langle w,w \rangle}$$ 
According to Cauchy-Schwarz inequality, we have:
$$-\Vert v \Vert \Vert w \Vert\leq \langle v, w \rangle \leq \Vert v \Vert \Vert w \Vert$$
$$- \Vert w \Vert \Vert v \Vert \leq \langle w, v \rangle \leq \Vert w \Vert \Vert v \Vert$$
$$\Rightarrow - \Vert w \Vert \Vert v \Vert - \Vert w \Vert \Vert v \Vert = -2\Vert v \Vert  \Vert w \Vert \leq \langle v, w \rangle + \langle w, v \rangle \leq \Vert v \Vert \Vert w \Vert+ \Vert w \Vert \Vert v \Vert = 2\Vert v \Vert \Vert w \Vert$$
Therefore, we can the above inequality is true, and thus the Triangle Inequality.



% %%%%%%%%%%%%%%%%%%%
\subsection*{P3.3}
% %%%%%%%%%%%%%%%%%%%
\textbf{Parallellogram law:  $\Vert v+w\Vert^2 + \Vert v-w\Vert^2 =2 (\Vert v\Vert^2+\Vert w\Vert^2)$}\\
We have the left-hand side of the equation:
\begin{align*}
\Vert v+w\Vert^2 + \Vert v-w\Vert^2 &= \langle v+w,v+w \rangle +  \langle v-w,v-w \rangle\\
& = \langle v,v+w \rangle + \langle w, v+w \rangle + \langle v, v-w \rangle - \langle w, v-w \rangle\\
& = \langle v, v \rangle + \langle v, w \rangle + \langle w, v \rangle + \langle w, w \rangle + \langle v, v \rangle - \langle v, w \rangle - \langle w, v \rangle + \langle w, w \rangle\\
&= 2\langle v,v \rangle + 2\langle w,w \rangle\\
&= 2(\Vert v \Vert^2 + \Vert w \Vert^2)
\end{align*}
which is equal to the right-hand side. We thus prove the Parallellogram law.


% %%%%%%%%%%%%%%%%%%%
\subsection*{P3.4}
% %%%%%%%%%%%%%%%%%%%
\textbf{Polarization identity:  $\left \langle v,w \right \rangle = \frac1{2}\big(\Vert v+w\Vert^2 -\Vert v\Vert^2 - \Vert w\Vert^2 \big)$}\\
We have the right-hand side of the equation:
\begin{align*}
\frac1{2}\big(\Vert v+w\Vert^2 -\Vert v\Vert^2 - \Vert w\Vert^2 \big) &= \frac1{2}\big(\langle v+w,v+w \rangle - \langle v,v \rangle - \langle w,w \rangle \big)\\
& = \frac1{2}\big(\langle v, v \rangle + \langle v, w \rangle + \langle w, v \rangle + \langle w,w \rangle - \langle v,v \rangle - \langle w,w \rangle \big)\\
& = \frac1{2}\big(\langle v,w \rangle + \langle w,v \rangle \big)\\
& = \frac1{2}\big(\Vert v\Vert\Vert w \Vert cos(v,w) + \Vert w \Vert \Vert v\Vert cos(w,v)\big)\\
& = \frac1{2}\big(2\Vert w \Vert \Vert v \Vert cos(v,w)\big)\\
& = \langle v,w \rangle
\end{align*}
which is equal to the left-hand side. We thus prove the Polarization identity.





% %%%%%%%%%%%%%%%%%%%
\subsection*{P3.5}
% %%%%%%%%%%%%%%%%%%%
\textbf{Find the distance from a point $\mathbf w=(w_x,w_y,w_z)^\top$ to the plane $H_n$ normal to $\mathbf n=(a,b,c)^\top$ and translated from the origin by vector $\mathbf p=(x_0,y_0,z_0)^\top$. Show that it is the smallest distance to any point in $H_n$.}\\
The equation of the plane $H_n$ is:
\begin{align*}
H_n &= a(x-x_0)\\
	&= a(x-x_0)+b(y-y_0)+c(z-z_0)\\
	&= ax + by + cz -(ax_0+by_0+cz_0)
\end{align*}

Distance from the point $w$ to the plane $H_n$ is:
$$D(w, H_n) = \frac{aw_x + b w_y + c w_z -(ax_0+by_0+cz_0)}{\sqrt{a^2+b^2+c^2}}$$
Assume for contradiction that there exists a point $w_{p1} \in H_n $ beside $w_p$ (the orthogonal projection of $w$ on $H_n$) whose length $ ww_{p1} < ww_p$. According to the Pythagorean theorem, we have:
$$ww_{p1}^2 = ww_p^2 + w_{p1}w_p^2$$
$$\Rightarrow ww_{p1} > ww_p$$
which contradicts the previous statement. Thus, $ww_p$ is the smallest distance to any point in $H_n$.



% %%%%%%%%%%%%%%%%%%%
\subsection*{P3.6}
% %%%%%%%%%%%%%%%%%%%
\textbf{Prove that $w_p=\mathsf{proj}_{V_k}(w)$ is the vector in subspace $V_k$ closest to $w$ using 2 approaches: }
\begin{enumerate}
\item \textbf{expand $\Vert u-v \Vert^2= \left \langle u-v,u-v \right \rangle$.}
\item \textbf{using Pythagorean theorem}
\end{enumerate}
In order to prove the above, we have to prove, for any $w_{p1} \in V$, $\Vert w-w_{p1} \Vert > \Vert w-w_{p}\Vert$.
\begin{enumerate}
\item expand $\Vert u-v \Vert^2= \left \langle u-v,u-v \right \rangle$.
\item using Pythagorean theorem\\
	 Let $a = w - w_{p}$, $b = w_p - w_{p1} \Rightarrow b + a = w - w_{p1}$. Since $a \perp b \Leftrightarrow \langle a,b\rangle = \langle b,a\rangle = 0$, we have the following:
	\begin{align*}
	\Vert w-w_{p1} \Vert^2 = \Vert b+a \Vert^2 &= \langle b+a,b+a \rangle\\
	&= \langle b,b\rangle + \langle b,a \rangle + \langle a,b\rangle + \langle a,a \rangle\\
	&= \Vert b \Vert^2 + 0 + 0 + \Vert a \Vert^2 > \Vert a \Vert^2 = \Vert w-w_p\Vert^2\\
	&\Rightarrow \Vert w-w_{p1} \Vert > \Vert w-w_p\Vert
	\end{align*}
We have thus proved the above using Pythagorean theorem.
\end{enumerate}

% %%%%%%%%%%%%%%%%%%%
\subsection*{P3.7}
% %%%%%%%%%%%%%%%%%%%
\textbf{Least squares solution of $A\mathbf x = \mathbf b$ is $\hat{\mathbf x}=\arg\min_{\mathbf x}\Vert \mathbf b - A\mathbf x\Vert$. By column view of matrix multiplication, $A\hat{\mathbf x}$ is the orthogonal projection of $\mathbf b$ onto the column space $\mathsf{Col}(A)$ of matrix $A$. Thus we must have $(\mathbf b - A\hat{\mathbf x})\perp\mathsf{Col}(A)$. Write down matrix form for this and solve for $\hat{\mathbf x}$, given that $A$ is full rank. Show that the orthogonal projection operator of $\mathbf b$ onto $\mathsf{Col}(A)$ is: $P_{\mathsf{Col}(A)} = A(A^{\top}A)^{-1}A^{\top}$.}\\
Assume that $\mathbf{v_1,v_2,\ldots,v_n}$ are the columns of $A$, we have $(\mathbf b - A\hat{\mathbf x})\perp\mathsf{Col}(A)$ in matrix form is:
$$[\mathbf{v_1, v_2,\ldots,v_n}]
\begin{bmatrix}
x_1\\
x_2\\
\vdots\\
x_n\\
\end{bmatrix}-
[\mathbf{v_1, v_2,\ldots,v_n}]
\begin{bmatrix}
\hat{x}_1\\
\hat{x}_2\\
\vdots\\
\hat{x}_n\\
\end{bmatrix}$$
We have that $\mathbf{b} = \mathbf{b}_{\mathsf{Col}(A)} + \mathbf{b}_{\mathsf{Col}(A)^\perp}$, and $\mathbf{b}_{\mathsf{Col}(A)} = A\hat{\mathbf{x}}$, so $\mathbf{b}-\mathbf{b}_{\mathsf{Col}(A)} = \mathbf{b}-A\hat{\mathbf{x}}$, which also equals to zero, because $\mathbf{b}-\mathbf{b}_{\mathsf{Col}(A)} =  \mathbf{b}_{\mathsf{Col}(A)^\perp} \in \mathsf{Col}(A)^\perp = N(\mathsf{Col}(A))$. We then have:

\begin{align*}
0 &= A^\mathsf{T}(\mathbf{b}-A\hat{\mathbf{x}}) = A^\mathsf{T}b - A^\mathsf{T}A\hat{\mathbf{x}}\\
&\Leftrightarrow A^\mathsf{T}\mathbf{b} = A^\mathsf{T}A\hat{\mathbf{x}}\\
&\Leftrightarrow \hat{\mathbf{x}} = (A^\mathsf{T}A)^{-1}A^\mathsf{T}\mathbf{b}
\end{align*}

% %%%%%%%%%%%%%%%%%%%
\subsection*{P3.8}
% %%%%%%%%%%%%%%%%%%%
\textbf{Show that the generalized dot product $\left \langle \mathbf x,\mathbf y \right \rangle_M = \mathbf x^\top M \mathbf y$ with $M\in\mathbb{R}^{n\times n}$ a symmetric positive definite matrix satisfies all 3 requirements of a proper inner product.}\\
Let $\mathbf x = \begin{bmatrix}
x_1\\
x_2\\
\vdots\\
x_n
\end{bmatrix}$,$\mathbf y = \begin{bmatrix}
y_1\\
y_2\\
\vdots \\
y_n
\end{bmatrix}$ and $M = \begin{bmatrix}
\mathbf m_1\\
\mathbf m_2\\
\vdots\\
\mathbf m_n

\end{bmatrix}$ with $\mathbf m_i$, $i=1,2,\ldots,n$, having $n$ elements. For any $\mathbf x,\mathbf x'$ in the same inner product space, $c \in R$, we have the following 3 requirements of an inner product:
\begin{enumerate}
\item $\langle \mathbf x +\mathbf x',y\rangle = \langle\mathbf x, \mathbf y\rangle + \langle\mathbf x',\mathbf y\rangle $ and $\langle c\mathbf x,\mathbf y\rangle = c\langle \mathbf x,\mathbf y\rangle$\\
	We have the following: 
	\begin{align*}
	\langle\mathbf x +\mathbf x',\mathbf y\rangle &= \begin{bmatrix}
	x_1+x_1' &
	x_2 + x_2' &
	\ldots &
	x_n+x_n'
	\end{bmatrix} M\mathbf y\\
&= \begin{bmatrix}
x_1\\
x_2\\
\vdots\\
x_n
\end{bmatrix}M\mathbf y + \begin{bmatrix}
x_1'\\
x_2'\\
\vdots\\
x_n'
\end{bmatrix}M\mathbf y\\
&=\langle\mathbf x,\mathbf y\rangle + \langle\mathbf x',\mathbf y\rangle
	\end{align*}
and \begin{align*}
\langle c\mathbf x, y\rangle &= \begin{bmatrix}
cx_1\\
cx_2\\
\vdots\\
cx_n
\end{bmatrix}My\\
&= c\begin{bmatrix}
x_1\\
x_2\\
\vdots\\
x_n
\end{bmatrix}My\\
&=c\langle\mathbf x,\mathbf y\rangle
\end{align*}
\item $\langle\mathbf x,\mathbf y\rangle = \langle\mathbf y,\mathbf x \rangle$\\
\begin{align*}
\langle \mathbf x,\mathbf y\rangle &= \begin{bmatrix}
x_1 &
x_2 &
\ldots &
x_n
\end{bmatrix}\begin{bmatrix}
\mathbf m_1\\
\mathbf m_2\\
\vdots\\
\mathbf m_n
\end{bmatrix}
\mathbf y\\
&= x_1\mathbf m_1 \mathbf y + x_2\mathbf m_2 \mathbf y + \ldots + x_n \mathbf m_n \mathbf y\\
&= y_1\mathbf m_1 \mathbf x + y_2\mathbf m_2 \mathbf x + \ldots + y_n \mathbf m_n \mathbf x\\
&= \begin{bmatrix}
y_1 &
y_2 &
\ldots &
y_n
\end{bmatrix}\begin{bmatrix}
\mathbf m_1\\
\mathbf m_2\\
\vdots\\
\mathbf m_n
\end{bmatrix}
\mathbf x\\
&= \langle \mathbf y,\mathbf x\rangle 
\end{align*}
\item $\langle \mathbf x,\mathbf x \rangle > 0 $\\
Since M is a positive definite matrix, it is obvious that:

$$\langle \mathbf x,\mathbf x \rangle = \mathbf x^\mathsf{T}M\mathbf x >0$$

\end{enumerate}
With the 3 requirements proved, we have the above.


% %%%%%%%%%%%%%%%%%%%
\subsection*{P3.9}
% %%%%%%%%%%%%%%%%%%%
\textbf{Prove that the null space and the row space of a matrix are orthogonal, i.e. every vector in null space is orthogonal to every vector in row space (zero dot product).}\\
Assume we have a matrix $A = \begin{bmatrix}
v_1\\
\vdots\\
v_n
\end{bmatrix}$, its null space $N(A)$ which contains vectors $x$ satisfying $Ax=0$, and its row space $R = span(v_1,\ldots,v_n)$. We have $Ax = 0 \Leftrightarrow v_1x=0, \ldots, v_nx=0$. Therfore, for $r \in R$, we have the following: $$rx = span(v_1,\ldots,v_n)x = a_1v_1x + \ldots+ a_nv_nx = 0$$
From here we have proven that $rx=0 \forall r \in R$, which means every $r$ is ortogonal with $x$, which proves what we have to prove.

Assume we have a matrix $A = \begin{bmatrix}
v_1\\
\vdots\\
v_n

\end{bmatrix}$, its null space $N(A)$ which contains vectors $x$ satisfying $Ax=0$, and its row space $R = span(v_1,\ldots,v_n)$. We have $Ax = 0 \Leftrightarrow v_1x=0, \ldots, v_nx=0$. Therfore, for $r \in R$, we have the following: $$rx = span(v_1,\ldots,v_n)x = a_1v_1x + \ldots+ a_nv_nx = 0$$
From here we have proven that $rx=0$, $\forall  r \in R$, which means every $r$ is ortogonal with $x$, which proves what we have to prove.




% %%%%%%%%%%%%%%%%%%%
\subsection*{P3.10}
% %%%%%%%%%%%%%%%%%%%
\textbf{}\\




% %%%%%%%%%%%%%%%%%%%
\section*{Section E}
% %%%%%%%%%%%%%%%%%%%
\subsection*{E3.1}
\textbf{Let $V$ be an $n$-dimensional inner product space with pairwise orthogonal subspaces
$$W_1, \dots, W_m\, ,$$
where $\sum_{i=1}^{m} \dim(W_i) = n$.
Prove that every vector $v \in V$ can be represented uniquely as
$$v = w_1 + \dots + w_m \, ,$$
where $w_i \in W_i$ for $i = 1, \dots, m$, i.e.,
$$V = W_1 \oplus \dots \oplus W_m \, .$$}\\

Assume that $V$ has basis vectors $\{v_1,v_2,\ldots , v_n\}$, and $W_1$ has bases $\{ v_1,v_2,\ldots, v_i\}$, $W_2$ has bases $\{v_{i+1}, v_{i+2},\ldots v_j\}$, \ldots , $W_m$ has bases $\{v_{k+1}, v_{k+2},\ldots , v_n \}$. We have any $v \in V$, $w_1 \in W_1$, $w_2\in W_2$,\ldots ,$w_m \in W_m$, for any $a_1,a_2,\ldots a_i, a_{i+1}, a_{i+2},$ $\ldots ,a_j,\ldots , a_{k+1}, a_{k+2}, \ldots ,a_n$ can be expressed as follow:
\begin{align*}
v &= a_1v_1 + a_2v_2 + \ldots + a_nv_n\\
w_1 &= a_1v_1 + a_2v_2 +\ldots +a_iv_i\\
w_2 &= a_{i+1}v_{i+1} + a_{i+2}v_{i+2} +\ldots + a_jv_j\\
\vdots\\
w_m &= a_{k+1}v_{k+1} + a_{k+2}v_{k+2}+\ldots + a_{n}v_{n}
\end{align*}
We then have:
\begin{align*}
w_1+w_2+\ldots+w_m = &a_1v_1 + a_2v_2 +\ldots +a_iv_i \\
&+a_{i+1}v_{i+1} + a_{i+2}v_{i+2} +\ldots + a_jv_j\\
&\ldots\\
&+a_{k+1}v_{k+1} + a_{k+2}v_{k+2}+\ldots + a_{n}v_{n}\\
&= v
\end{align*}
since $\sum_{i=1}^{m} \dim(W_i) = n$. Therefore $v = w_1+w_2+\ldots+w_m$ uniquely, for an unique set of $a_1, a_2,\ldots, a_n$.



\subsection*{E3.2}
\textbf{Prove again for 2D geometry that $\cos\theta =\frac{\mathbf a^\top\mathbf b}{\Vert \mathbf a\Vert \Vert \mathbf b\Vert}$ using only cosine (and sine) laws.}
\begin{align*}
cos(\theta) = cos(\beta - \alpha) &=cos(\beta)cos(\alpha)+sin(\beta)sin(\alpha)\\
&= \frac{b_1}{\Vert \mathbf b \Vert}\frac{a_1}{\Vert\mathbf a \Vert} + \frac{b_2}{\Vert\mathbf b \Vert}\frac{a_2}{\Vert\mathbf a \Vert} \\
&= \frac{b_1a_1}{\Vert\mathbf b\Vert \Vert\mathbf a\Vert} +\frac{b_2a_2}{\Vert\mathbf b\Vert \Vert\mathbf a\Vert}\\
&= \frac{b_1a_1+b_2a_2}{\Vert\mathbf b\Vert \Vert\mathbf a \Vert}\\
&= \frac{\begin{bmatrix}
a_1 & a_2
\end{bmatrix}•
\begin{bmatrix}
b_1\\
b_2
\end{bmatrix}}
{\Vert\mathbf b\Vert \Vert\mathbf a \Vert}\\
&= \frac{\mathbf a^\top\mathbf b}{\Vert \mathbf a\Vert \Vert \mathbf b\Vert}
\end{align*}



\subsection*{E3.3}
\textbf{Given an orthonormal basis $\mathcal{U}=(u_1,\dots,u_n)$ of $\mathbb{R}^n$. Let $U$ be the matrix whose distinct columns are the basis vectors in $\mathcal{U}$. Prove again $U$ is a transformation that does not change distance, angle, size of the objects being transformed: geometrically it is a rotation or reflection!}\\
Let $A$ be the object being transformed, and $A = [\mathbf{a}_1,\mathbf{a}_2,\ldots,\mathbf{a}_n]$. Since $U$ is a matrix consisting of orthonormal vector $\Leftrightarrow U^{\mathsf{T}}U=UU^{\mathsf{T}}=I_n$, then we have the following:
$$\langle U\mathbf{a}_i,U\mathbf{a}_j\rangle = U\langle \mathbf{a}_i,U\mathbf{a}_j\rangle=\langle \mathbf{a}_i;U^\mathsf{T}U\mathbf{a}_j\rangle = \langle \mathbf{a}_i,\mathbf{a}_j\rangle$$
for any $i,j=1,2,\ldots,n$. We can conclude that the distance does not change after transforming.\\
We can prove that the angle and the size does not change by the following: 
$$\Vert U\mathbf a_i\Vert^2 = \langle U\mathbf a_i, U\mathbf a_i \rangle = \langle \mathbf a_i,\mathbf a_i \rangle = \Vert \mathbf a_i \Vert^2$$
$$cos(U\mathbf a_i,U\mathbf a_j)=\frac{\langle U\mathbf a_i, U\mathbf a_j\rangle}{\Vert U\mathbf a_i \Vert \Vert U\mathbf a_j \Vert } = \frac{\langle \mathbf a_i,\mathbf a_j \rangle}{\Vert \mathbf a_i \Vert \Vert \mathbf a_j \Vert } = \cos(\mathbf a_i,\mathbf a_j)$$
		



\newpage 






\end{document}
